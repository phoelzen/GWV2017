\documentclass[12pt, paper=a4]{article}
\usepackage[utf8]{inputenc}
\usepackage[german]{babel}
\usepackage{amsmath}
\usepackage{amssymb}
\usepackage{listings}
\usepackage{graphicx}
\usepackage{fancyhdr}

\author{Benjamin Cordt, xxxxxxx\\Paul Hölzen, 6673477\\Ino xx, xxxxxxx}

\title{GWV Hausaufgaben 1}

\rhead{B. Cordt, P. Hölzen, I. xx}
\pagestyle{fancy}
\begin{document}
\maketitle

\section*{Aufgabe 1.1}
\begin{itemize}
\item Fully observable $\Leftrightarrow$ partially observable\\
      Eine AI Applikation muss unter Berücksichtigung der Beobachtbarkeit der Welt entworfen werden, da
      nicht beobachtbare Komponenten einen großen Einfluss auf die Umwelt und damit auf das Verhalten der
      AI haben können bzw. sollten.\\
      Ein Problem, das in einer nur teilweise beobachtbaren Welt besteht, nicht aber in einer vollständig
      beobachtbaren, ist das mögliche Übersehen oder falsche Behandeln von unsichtbaren Einflüssen.
\item Discrete $\Leftrightarrow$ continuous\\
      Wenn von einer diskreten Welt ausgegangen wird, obwohl eine kontinuierliche vorliegt, können die
      ``Zwischenwerte'' nicht abgebildet oder verarbeitet werden.\\
      Ein Problem einer kontinuierlichen Welt gegenüber einer diskreten ist die Kathegorisierung von
      Ereignissen oder Werten, da diese zwischen Erwartungs-/Schwellwerten liegen können.
\item Deterministic $\Leftrightarrow$ stochastic
      In einer deterministischen Welt hat jede Aktion geau ein festes Ergebnis. Eine AI die mit diesem
      Konzept entworfen wurde, stößt in einer stochastischen Welt auf Probleme, sobald dieses Ergebnis
      abweicht.\\
      Dies ist auch das Problem einer stochastischen Welt im Gegensatz zur deterministischen: Reaktionen
      auf die Umwelt und müssen sehr flexibel sein um auch unvorhergesehene Ergebnisse verarbeiten zu
      können.
\end{itemize}

\end{document}