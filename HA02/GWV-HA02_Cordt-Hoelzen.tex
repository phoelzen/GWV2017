\documentclass[12pt, paper=a4]{article}
\usepackage[utf8]{inputenc}
\usepackage[german]{babel}
\usepackage{amsmath}
\usepackage{amssymb}
\usepackage{listings}
\usepackage{enumitem}
\usepackage{graphicx}
\usepackage{fancyhdr}

\author{Benjamin Cordt\\Paul Hölzen}

\title{GWV Hausaufgaben 2}

\rhead{B. Cordt, P. Hölzen}
\pagestyle{fancy}
\begin{document}
\maketitle

\section*{Aufgabe 2.4}
\begin{itemize}
\item blabla
\item $state: <(c_1, w_1), ..., (c_n, w_n)>$ wobei es eine Menge an Komponenten des Hauses gibt,
      deren Elemente als Tupel mit Elementen einer Menge an Arbeitern kombiniert werden können.\\
      Seien z.B. die Komponenten eines Hauses\\
      $C=\{floor, wall 1, wall 2, wall 3, wall 4, ceiling, roof, paint 1, paint 2,
      paint 3, paint 4\}$ und eine Menge von Arbeitern $W=\{a, b, c\}$.
      Ein Beispielzustand wäre dann $<(floor, a), (wall 1, b), (wall 2, c)>$.\\
\end{itemize}
\end{document}
