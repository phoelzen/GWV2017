\documentclass[12pt, paper=a4]{article}
\usepackage[utf8]{inputenc}
\usepackage[german]{babel}
\usepackage{amsmath}
\usepackage{amssymb}
\usepackage{listings}
\usepackage{enumitem}
\usepackage{graphicx}
\usepackage{fancyhdr}

\author{Benjamin Cordt\\Paul Hölzen}

\title{GWV Hausaufgaben 2}

\rhead{B. Cordt, P. Hölzen}
\pagestyle{fancy}
\begin{document}
\maketitle

\section*{Aufgabe 2.4}
\begin{itemize}
\item $state: <P,F>$ wobei $P$ eine Matrix ist, deren Koordinaten Orte in der Wohnung repräsentieren.
      Die Werte sind dabei Element der Menge $\tilde{F} = F \cup x \cup 0$.
      $F$ ist die Menge von Möbeln, die Werte $x$ und $0$ stehen für Plätze auf denen keine
      Möbel stehen können und freie Stellen.\\
      Die Menge $F$ besteht aus Zwei-Tupeln mit einem eindeutigen Bezeichner $b$ an der ersten
      und zwei Zahlen $x, y \in \mathbb{N}$ an der zweiten und dritten Stelle, die die benötigte
      Fläche beschreiben.\\
      
      Der Startzustand besteht dann aus $P$, einer Matrix, die eine leere Wohnung darstellt, und
      $F$ der Menge aller Möbel, die darin aufgestellt werden sollen. Bei einem Zustandsübergang
      wird ein Element $f$ aus $F$ entfernt und benachbarte Werte $0$ der Matrix $P$ werden
      entsprechend der Maße $x$ und $y$, auf den Bezeichner $b$ gesetzt.\\
      Gültige Zustände seien nur solche, bei denen Stühle benachbart zu Tischen sind und Platz
      vor der Tür bleibt etc. Das Ziel ist erreicht, wenn die Menge $F$ leer ist. Das bedeutet es
      gibt vielleicht keinen Zielzustand, wenn z.B. zu viele Möbel für eine kleine Wohnung eingeplant
      wurden oder alles voller Türen ist.\\
      Der am besten geeignete Suchalgorithmus ist die Breitensuche, da viele Teilbäume kein Ziel
      enthalten werden und die Tiefensuche hier potentiell viel Zeit verschwendet.

\item $state: <(c_1, w_1), ..., (c_n, w_n)>$ wobei es eine Menge an Komponenten des Hauses gibt,
      deren Elemente als Tupel mit Elementen einer Menge an Arbeitern kombiniert werden können.\\
      Seien z.B. die Komponenten eines Hauses\\
      $C=\{floor, wall_1, wall_2, wall_3, wall_4, ceiling, roof, paint_1, paint_2,
      paint_3, paint_4\}$ und eine Menge von Arbeitern $W=\{a, b, c\}$.
      Ein Beispielzustand wäre dann $<(floor, a), (wall 1, b), (wall 2, c)>$.\\
      Die zeitlichen Abhängigkeiten der Komponenten sind wie folgt:\\
	  \begin{align*}       
      &a       && \underset{\text{needs}}{\longrightarrow} && b\\
      &paint_n && \longrightarrow && wall_n\\
      &ceiling && \longrightarrow && wall_1 \wedge wall_2 \wedge wall_3 \wedge wall_4\\
      &roof    && \longrightarrow && ceiling
      \end{align*}
      Dadurch sind bestimmte Zustände oder Zustandsfolgen nicht möglich.\\
      Gesucht wird der kürzeste Pfad zum Ziel (alle Komponenten sind fertig gebaut), also ist die
      Breitensuche die beste Wahl.

\item ...
\end{itemize}
\end{document}
