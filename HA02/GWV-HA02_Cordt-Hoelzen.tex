\documentclass[12pt, paper=a4]{article}
\usepackage[utf8]{inputenc}
\usepackage[german]{babel}
\usepackage{amsmath}
\usepackage{amssymb}
\usepackage{listings}
\usepackage{enumitem}
\usepackage{graphicx}
\usepackage{fancyhdr}
\setlength{\parindent}{0in}

\author{Benjamin Cordt\\Paul Hölzen}

\title{GWV Hausaufgaben 2}

\rhead{B. Cordt, P. Hölzen}
\pagestyle{fancy}
\begin{document}
\maketitle

\section*{Aufgabe 2.3}

\begin{itemize}
\item Es gibt ein Spielfeld mit eindeutigen Positionen und dort
verwendbaren Verkehrsmitteln
\item Mr X kann sich vor und zurück bewegen,
\item aber nur Verkehrsmittel nutzen, wenn er ein Ticket hat
\item Tickets:
    \begin{itemize}
    \item 4 Taxi.
    \item 3 Bus,
    \item 3 Ubahn
    \end{itemize}
\item (Black Tickets zur Vereinfachung weggelassen)
\end{itemize}

Sonstige Annahmen:\\
Mr X kann sich A Steps bewegen. (und nur diese)\\
Goal: So weit wie möglich von Detektiven weg,(optional wäre möglich
auch deren möglichen Positionen nach den ziehen zu berücksichtigen,
hier Bezug zu ihren Startpunkten)\\
Answer:\\
State: $< X_p , DP_n , T, Tickets, S, DS_n , D_n >$\\
Goal: $<?, DP_n ,?,?,?,?,D_n > ; mit D_n >= DS_n mit D_n > DS_n \rightarrow max$\\

\begin{itemize}
\item $X_p$ : Position von Mr. X zum Start die Startposition
\item $DP_n$ : Positionen der Detektive $<DP_1 , .... , DP_n >$ für Anzahl der Detektive $n$
\item $T$: Mögliche Transportmittel an einer Position zu einer bestimmten anderen $< Taxi, Bus, U-Bahn>$ mit Elementen $(0,1)$ verfügbar und nicht verfügbar
\item $Tickets$: $<T_{Taxi} , T_{Bus} , T_{U-Bahn} >$ wobei die Tikets Anfangs feststehen und verbraucht werden
\item $S$: Steps von Mr X (reduzieren sich bei Bewegung)
\item $DS_n$ : Entfernung Mr. X zu den Detektiven zum Start $< DS_1 , ..., DS_n >$ für Anzahl der Detektive $n$
\item $D_n$ : Aktuelle Entfernung Mr. X zu den Detektiven $< D_1 , ..., D_n >$ für Anzahl der Detektive $n$ zum Start = $DS_n$
\end{itemize}

Übergangsregeln: Pfad auswählen, der am weitesten Weg von den
Detektiven ist (oder notfalls gleich)

\begin{itemize}
\item While Mr X hat Steps $A>0$
    \begin{itemize}
    \item If möglich: Choose mögliche Nachbarposition mit für alle Elemente $Dn> DS_n$ Anzahl der Detektive $n$
        \begin{itemize}
        \item move MR X to new Position
        \item remove used Ticket
        \item decrease Step by 1
        \end{itemize}
    \end{itemize}
\item else
    \begin{itemize}
    \item If möglich: Choose mögliche Nachbarposition für ein $D_n > DS_n$ and new Position $\neq DP_n$
        \begin{itemize}
        \item move Mr. X to new Position
        \item remove used Ticket
        \item decrease Step by 1
        \end{itemize}
    \item else
        \begin{itemize}
        \item hold Position and terminiert
        \end{itemize}
    \end{itemize}
\end{itemize}

Zustandsänderungen: von einem Zustand zum nächsten reduziert sich
die Anzahl der Steps und Tickets. Die möglichen Transportmittel
können sich ändern und die Position von Mr. X als auch die Distanz
ändern sich. Die Startdistanz und Position der Detektive bleibt bei uns
immer gleich.\\
Hier : Ist Best-first Suche in Abwandlung am besten geeignet.\\



\section*{Aufgabe 2.4}
\begin{itemize}
\item $state: <P,F>$ wobei $P$ eine Matrix ist, deren Koordinaten Orte in der Wohnung repräsentieren.
    Die Werte sind dabei Element der Menge $\tilde{P} = F \cup x \cup 0$.
      $F$ ist die Menge von Möbeln, die Werte $x$ und $0$ stehen für Plätze auf denen keine
      Möbel stehen können und freie Stellen.\\
      Die Menge $F$ besteht aus Zwei-Tupeln mit einem eindeutigen Bezeichner $b$ an der ersten
      und zwei Zahlen $x, y \in \mathbb{N}$ an der zweiten und dritten Stelle, die die benötigte
      Fläche beschreiben.\\
      
      Der Startzustand besteht dann aus $P$, einer Matrix, die eine leere Wohnung darstellt, und
      $F$ der Menge aller Möbel, die darin aufgestellt werden sollen. Bei einem Zustandsübergang
      wird ein Element $f$ aus $F$ entfernt und benachbarte Werte $0$ der Matrix $P$ werden
      entsprechend der Maße $x$ und $y$, auf den Bezeichner $b$ gesetzt.\\
      Gültige Zustände seien nur solche, bei denen Stühle benachbart zu Tischen sind und Platz
      vor der Tür bleibt etc. Das Ziel ist erreicht, wenn die Menge $F$ leer ist. Das bedeutet es
      gibt vielleicht keinen Zielzustand, wenn z.B. zu viele Möbel für eine kleine Wohnung eingeplant
      wurden oder alles voller Türen ist.\\
      Der am besten geeignete Suchalgorithmus ist die Breitensuche, da viele Teilbäume kein Ziel
      enthalten werden und die Tiefensuche hier potentiell viel Zeit verschwendet.

\item $state: <(c_1, w_1), ..., (c_n, w_n)>$ wobei es eine Menge an Komponenten des Hauses gibt,
      deren Elemente als Tupel mit Elementen einer Menge an Arbeitern kombiniert werden können.\\
      Seien z.B. die Komponenten eines Hauses\\
      $C=\{floor, wall_1, wall_2, wall_3, wall_4, ceiling, roof, paint_1, paint_2,
      paint_3, paint_4\}$ und eine Menge von Arbeitern $W=\{a, b, c\}$.
      Ein Beispielzustand wäre dann $<(floor, a), (wall 1, b), (wall 2, c)>$.\\
      Die zeitlichen Abhängigkeiten der Komponenten sind wie folgt:\\
	  \begin{align*}       
      &a       && \underset{\text{needs}}{\longrightarrow} && b\\
      &paint_n && \longrightarrow && wall_n\\
      &ceiling && \longrightarrow && wall_1 \wedge wall_2 \wedge wall_3 \wedge wall_4\\
      &roof    && \longrightarrow && ceiling
      \end{align*}
      Dadurch sind bestimmte Zustände oder Zustandsfolgen nicht möglich.\\
      Gesucht wird der kürzeste Pfad zum Ziel (alle Komponenten sind fertig gebaut), also ist die
      Breitensuche die beste Wahl.

\item $state: <x, f_1, ..., f_n>$ wobei $f_n \in \mathbb{N}^0 \cup -1$ und $n$ die Anzahl der
      Stockwerke ist. In jedem $f_n$ ist die maximale Wartezeit der Personen codiert,
      $-1$ bedeutet dass niemand wartet. $x$ ist das aktuelle Stockwerk des Fahrstuhls. Ein
      Zustandsübergang bewegt den Fahrstuhl um ein Stockwerk nach oben oder unten ($x +- 1$) und
      es vergeht eine Zeiteinheit ($\forall f_n: n \neq x: f_n + 1$).\\
      Wenn $f_n:n=x \wedge f_n > 0$ nimmt der Fahrstuhl die wartenden Gäste mit und $f_n := -1$.
      Wenn ein Fahrgast im Fahrstuhl ist wählt er ein Stockwerk $n$ zu dem er fahren will. Ist
      $f_n > 0$ dann bleibt es unverändert, ist es $-1$ wird es auf $0$ gesetzt.\\
      
      Der entsthende Graph ist bei immer wieder nachkommenden Fahrgästen unendlich. Das Ziel ist
      das Minimum aller $f_n$ zu erreichen. Dafür kann die Breitensuche verwendet werden.
\end{itemize}
\end{document}
